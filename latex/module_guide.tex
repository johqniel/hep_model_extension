\section{Concept}
A "Module" in this framework is a self-contained behavior that can be applied to agents (e.g., Movement, Aging, Mating). Modules are written in Fortran and exposed to the scheduling logic via a standardized interface.

\section{Step-by-Step Guide}

\subsection{1. Implement the Logic (Fortran)}
Create a new subroutine in \texttt{src/simulation\_modules/mod\_modules.f95} (or a new file). the subroutine must accept an \texttt{Agent} pointer.

\begin{lstlisting}[language=Fortran]
subroutine my_new_behavior(self)
    use mod_agent_world
    implicit none
    class(Agent), intent(inout) :: self
    
    ! Implement logic here
    self%pos_x = self%pos_x + 0.1
    
end subroutine my_new_behavior
\end{lstlisting}

\subsection{2. Register the Module ID (Python Interface)}
In \texttt{src/interfaces/python\_interface.f95}, define a new integer constant for your module ID.

\begin{lstlisting}[language=Fortran]
integer, parameter :: MODULE_MY_BEHAVIOR = 9
\end{lstlisting}

Then, add a case to the \texttt{step\_simulation} dispatch loop:

\begin{lstlisting}[language=Fortran]
select case (active_module_ids(jp))
    ! ... existing cases
    case (MODULE_MY_BEHAVIOR)
        call apply_module_to_agents(my_new_behavior, t)
end select
\end{lstlisting}

\subsection{3. Expose to Python (UI)}
In \texttt{python/spawn\_editor.py}, add your new module to the \texttt{available\_modules} dictionary making sure the ID matches the Fortran constant.

\begin{lstlisting}[language=Python]
self.available_modules = {
    # ...
    "My New Behavior": 9
}
\end{lstlisting}

\subsection{4. Compilation}
Run the build script to recompile the extension:
\begin{lstlisting}[language=Bash]
./build_extension.sh
\end{lstlisting}

\section{Best Practices}
\begin{itemize}
    \item \textbf{Statelessness}: Try to keep modules stateless where possible. Use agent variables to store state.
    \item \textbf{Grid Access}: Use \texttt{self\%grid} to access neighbor information.
    \item \textbf{Performance}: Avoid heavy I/O operations inside module loops.
\end{itemize}
