\chapter{Fortran Only Version}

The simulation framework includes a standalone Fortran executable, allowing users to run the model without the Python interface or GUI. This is particularly useful for high-performance computing (HPC) environments, batch processing, or profiling.

\section{Building the Executable}
The standalone version is built using the provided shell script \texttt{build\_fortran.sh}. This script compiles all necessary modules, creates a static library (\texttt{libhep.a}), and links the final executable.

\subsection{Prerequisites}
\begin{itemize}
    \item \textbf{Compiler}: \texttt{gfortran} (or compatible Modern Fortran compiler).
    \item \textbf{Libraries}: NetCDF and NetCDF-Fortran development libraries (\texttt{libnetcdf-dev}, \texttt{libnetcdff-dev}).
\end{itemize}

\subsection{Build Command}
To build the project, run the following command from the project root:
\begin{lstlisting}[language=bash]
./build_fortran.sh
\end{lstlisting}

This will create:
\begin{enumerate}
    \item A \texttt{build/} directory containing compiled object files (\texttt{.o}) and module files (\texttt{.mod}).
    \item A static library \texttt{build/libhep.a}.
    \item The standalone executable \texttt{build/main\_fortran}.
\end{enumerate}

\section{Running the Simulation}
The executable can be run directly from the command line. It reads the specific configuration file located at \texttt{input/config/main\_fortran\_config.nml}.

\subsection{Execution}
\begin{lstlisting}[language=bash]
./build/main_fortran
\end{lstlisting}

\subsection{Command Line Arguments}
The standalone version supports command-line arguments to override specific runtime parameters:

\begin{itemize}
    \item \texttt{--output\_interval <integer>}: Sets the interval (in ticks) at which simulation status is printed to the console. Default is 1000.
\end{itemize}

\textbf{Example:}
\begin{lstlisting}[language=bash]
./build/main_fortran --output_interval 500
\end{lstlisting}

\section{Configuration}
Unlike the Python version which uses \texttt{basic\_config.nml}, the standalone Fortran version hardcodes the configuration path to:
\begin{center}
    \texttt{input/config/main\_fortran\_config.nml}
\end{center}

Ensure this file exists and contains valid namelists (\texttt{\&dims} and \texttt{\&config}) before running. The structure of this file is identical to the standard configuration files used by the GUI.
